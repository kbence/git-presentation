\documentclass[display,t]{beamer}

\usetheme{Szeged}
\usefonttheme[onlylarge]{structurebold}
\setbeamerfont*{frametitle}{size=\normalsize,series=\bfseries}
\setbeamertemplate{navigation symbols}{}

\usepackage[hungarian]{babel}
\usepackage[utf8]{inputenc}
\usepackage{times}
\usepackage[T1]{fontenc}
\usepackage{listings}
\usepackage{graphicx}
\usepackage{tikz}
\usetikzlibrary{positioning,shapes,shadows,arrows}

\title{Git init}
\subtitle{Bevezetés a Git használatába}
\author{Kiglics Bence}

\begin{document}


\tikzstyle{commit} = [rectangle, draw = black, fill = white, rounded corners,
                      inner sep = 4pt, outer sep = 1pt, text centered,
                      text width = 0.8cm, font=\small]
\tikzstyle{current commit} = [commit, fill = yellow]
\tikzstyle{helper node} = [text width = 0.8cm, inner sep = 4pt, outer sep = 1pt]


\frame{\titlepage}

\begin{frame}{Tartalomjegyzék}
    \tableofcontents
\end{frame}

\section{Bevezetés}
\subsection{A verziókövető rendszerek rövid története}
\subsubsection{Korai verziókövetők}

\begin{frame}{Korai verziókövetők}
    \pause
    \begin{block}{SCCS (Source Code Control System)}
        \begin{itemize}
            \pause \item UNIX fejlesztés
            \pause \item Első tapasztalatok
        \end{itemize}
    \end{block}
    \pause
    \begin{block}{RCS (Revision Control System)}
        \begin{itemize}
            \pause \item Csak egyedi fájlokon dolgozik
            \pause \item Diff/patch
            \pause \item Branching
        \end{itemize}
    \end{block}
\end{frame}

\subsubsection{Központosított rendszerek}

\begin{frame}{Központosított rendszerek}
    \pause
    \begin{block}{CVS (Concurrent Versioning System)}
        \begin{itemize}
            \pause \item RCS utódja
            \pause \item Cliens/szerver
        \end{itemize}
    \end{block}
    \pause
    \begin{block}{SVN (Subversion)}
        \begin{itemize}
            \pause \item CVS lecserélésére
            \pause \item HTTP támogatása
        \end{itemize}
    \end{block}
\end{frame}

\subsubsection{Elosztott rendszerek}

\begin{frame}{Elosztott rendszerek}
    \pause
    \begin{block}{BZR (Bazaar)}
        \begin{itemize}
            \pause \item A Canonical Inc. fejlesztése
            \pause \item Launchpad
        \end{itemize}
    \end{block}
    \pause
    \begin{block}{HG (Mercurial)}
        \begin{itemize}
            \pause \item Linux kernel
            \pause \item Binary diff
        \end{itemize}
    \end{block}
\end{frame}

\begin{frame}{Elosztott rendszerek}
    \pause
    \begin{block}{Git}
        \begin{itemize}
            \pause \item Linus torvalds
            \pause \item A BitKeeper fizetőssé vált
            \pause \item Diff helyett teljes fájlok tárolása
            \pause \item Kis projektekre és hatalmas kódbázisok követésére egyaránt alkalmas
        \end{itemize}
    \end{block}
\end{frame}

\section{A verziókövetés alapjai}

\begin{frame}
    Egy repository felépítése \pause
    
    \begin{center}
        \begin{tikzpicture}[->, node distance=0.4cm]
            \node (C0) [commit] {C0};
            \pause
            
            \node (C1) [commit, right=of C0] {C1};
            \draw (C0) -> (C1);
            \pause
            
            \node (C2) [commit, right=of C1] {C2};
            \draw (C1) -> (C2);
            \pause
            
            \node (C1to3) [helper node, below=of C1] {};
            \node (C3) [commit, right=of C1to3] {C3};
            \draw (C1) -> (C3);
            \pause
            
            \node (C4) [commit, right=of C2] {C4};
            \draw (C2) -> (C4);
            \pause
            
            \node (C5) [commit, right=of C3] {C5};
            \draw (C3) -> (C5);
            \pause
            
            \node (C6) [commit, right=of C4] {C6};
            \draw (C4) -> (C6);
            \draw (C5) -> (C6);
        \end{tikzpicture}
    \end{center}
\end{frame}

\subsection{Lokális verziókövetés Git segítségével}


\end{document}

